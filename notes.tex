\documentclass{article} 
\usepackage{tikz-cd}
\usepackage{amssymb}
\usepackage{bm}
\usepackage{amsmath}

\newcommand{\myparagraph}[1]{\paragraph{#1}\mbox{}\\}
\newcommand{\cat}[1]{\mathcal{#1}} % category syntax
\newcommand{\cato}[1]{\cat{#1}_0} % objects of category
\newcommand{\catm}[1]{\cat{#1}_1} % morphisms of category
\newcommand{\ob}[1]{\mathsf{#1}} % category object syntax
\newcommand{\scat}{\cat{C}\ob{at}}
\newcommand{\catop}[1]{\cat{#1}^{op}}
\newcommand{\catopo}[1]{(\catop{#1})_0}
\newcommand{\catopm}[1]{(\catop{#1})_1}
\newcommand{\all}{\enspace \forall}
\newcommand{\allin}[2]{\all #1 \in #2}
\newcommand{\limc}[1]{\bm{Lim}#1}
\newcommand{\cata}[1]{\mathsf{cata} \thinspace #1}

\newtheorem{theorem}{Theorem}
\newtheorem{lemma}[theorem]{Lemma}
\newtheorem{proposition}[theorem]{Proposition}
\newtheorem{corollary}[theorem]{Corollary}
\newtheorem{definition}[theorem]{Definition}
\newtheorem{example}[theorem]{Example}
\newtheorem{remark}[theorem]{Remark}
\newtheorem{compositionlaw}[theorem]{Composition Law}
\newtheorem{alg}[theorem]{Algorithm}



\begin{document}

\title {Category theory notes}
\maketitle

\section{Definition of a category}
\begin{definition}
	A category $\cat{C}$ has objects $\ob{a}$ and arrows $f: \ob{a} \rightarrow \ob{b}$ between objects.
	$\cato{C}$ denotes the objects of catogory $\cat{C}$ and $\catm{C}$ denotes its arrows.
	For an arrow $f: \ob{a} \rightarrow \ob{b}$ the object $\ob{a}$ is called its domain and $\ob{b}$ its codomain: also denoted $\ob{a} = dom(f)$ and $\ob{b} = cod(f)$.
	A category $\cat{C}$ must adhere to the following \textbf{axioms}:
	\begin{itemize}
		\item if $f$ and $g$ are two functions such that $cod(f)=dom(g)$, then the composition $gf$ (or $g.f$) is a function from $dom(f)$ to $cod(g)$:

		\begin{tikzcd}
			(\ob{a} \arrow[r,"f"] & \ob{b} \arrow[r,"g"] & \ob{c}) \quad {\displaystyle \mapsto} \quad ( \ob{a} \arrow[r, "gf"] & \ob{c})
		\end{tikzcd}
		\item composition of arrows is \textbf{associative}: i.e. given $f: \ob{a} \rightarrow \ob{b}$, $g: \ob{b} \rightarrow \ob{c}$ and $h: \ob{c} \rightarrow \ob{d}$, then $(hg)f = h(gf): \ob{a} \rightarrow \ob{d}$
		\item every element $\ob{a} \in \cato{C}$ has an identity arrow $id_{\ob{a}}: \ob{a} \rightarrow \ob{a}$ satisfying $id_{\ob{a}}.f = f \allin{f}{\catm{C}}$ with $cod(f)=\ob{a}$ and
		$g.id_{\ob{a}} = g \allin{g}{\catm{C}}$ with $dom(g)=\ob{a}$.
	\end{itemize}
\end{definition}


\begin{example}

	A \textbf{preorder} is a set $X$ together with a binary relation $\leq$ which is
	\begin{itemize}
		\item reflexive: i.e. $x \leq x \allin{x}{X} $
		\item transitive: i.e. $x \leq y, y \leq z \Rightarrow x \leq z \allin{x,y,z}{X}$
	\end{itemize}
	any such a preorder can be seen as a category $\cat{C}$ with elements being the objects of $X$ and a unique arrow $x \rightarrow y$ iff $x \leq y$.

\end{example}

\begin{example}
	A \textbf{monoid} is a set $X$ with a binary operation, written like multiplication $xy$ for $x,y \in X$ which is associative and has a unit element $e \in X$ such that $ex = xe = x \allin{x}{X}$.
	A monoid can be interpreted as a category with one object and one arrow $x$ for every $x \in X$.
\end{example}

\begin{example}
	$Top$, $Grp$, $Mon$, $Rng$, $Grph$, $Pos$ are the categories with \textbf{topological spaces, groups, monoids, rings, graphs, posets} as objects and their respective homomorphisms (structure preserving functions) as arrows.
	$Set$ is the category of sets with functions between sets as arrows.
\end{example}

\begin{definition}
	A category $\cat{C}$ is called \textbf{small} iff both $\cato{C}$ and $\catm{C}$ are sets. The category is called \textbf{locally small}
	iff for any two objects $\ob{a},\ob{b} \in \cat{C}$ the morphisms between $\ob{a}$ and $\ob{b}$ (denoted $Hom(\ob{a},\ob{b})$) form a set.
\end{definition}

\begin{definition}
	Let $\cat{C}$ be a category, the \textbf{oposite category} is denoted $\catop{C}$. This category has the same objects, but arrows pointing the other way.
	I.e. if $f: \ob{a} \rightarrow \ob{b}$ in $\catm{C}$, then $\bar{f}: \ob{b} \rightarrow \ob{a}$ in $\catopm{C}$. Composition of arrows is defined as $\bar{f}\bar{g} = \bar{gf}$.
\end{definition}

\begin{definition}
	An arrow $f: \ob{a} \rightarrow \ob{b}$ is called a \textbf{monomorphism} if for any other object $\ob{c}$ and morphisms $g,h: \ob{c} \rightarrow \ob{a}$ $fg = fh$ implies $g=h$.
	The arrow is called \textbf{epimorphism} if for any object $\ob{c}$ and morphisms $g,h: \ob{b} \rightarrow \ob{c}$ $gf = hf$ implies $g=h$.
\end{definition}

\begin{lemma}
	Monomorphisms in $Set$ correspond to injective functions, where epimorphisms in $Set$ correspond to surjective functions.
\end{lemma}

\begin{lemma}
	If $gf$ is mono, then $f$ is mono, so by duality: if $fg$ is epi, then $f$ is epi.
\end{lemma}

\begin{definition}
	An epi $f: \ob{a} \rightarrow \ob{b}$ is called \textbf{split epi} if $\exists g: \ob{b} \rightarrow \ob{a}$ such that $fg = id_{\ob{b}}$.
	Dually a mono $f: \ob{a} \rightarrow \ob{b}$ is called \textbf{split mono} if $\exists g: \ob{b} \rightarrow \ob{a}$ such that $gf = id_{\ob{a}}$.
\end{definition}

\begin{definition}
	A morphism $f: \ob{a} \rightarrow \ob{b}$ is called an \textbf{isomorphism} if $\exists g: \ob{b} \rightarrow \ob{a}$ such that $fg = id_{\ob{b}}$ and and $gf = id_{\ob{a}}$.
	In this case the objects $\ob{a}$ and $\ob{b}$ are called \textbf{isomorphic}.
\end{definition}

\begin{lemma}
	\begin{itemize}
		\item if two of $f, g$ and $fg$ are iso, then so is the third
		\item if $f$ is epi and split mono, it is iso
		\item if $f$ is split epi and mono, it is iso
	\end{itemize}
\end{lemma}

\begin{definition}
	An object $\ob{a} \in \cato{C}$ is called a \textbf{terminal object} if for any other object $\ob{b} \in \cato{C}$ there is a unique arrow $f: \ob{b} \rightarrow \ob{a}$.
	(E.g. singletons are terminal objects in $Set$) Similarly an object $\ob{a} \in \cato{C}$ is called a \textbf{initial object} if for ano other object $\ob{b} \in \cato{C}$ there is a unique arrow $f: \ob{a} \rightarrow \ob{b}$.
	(E.g. the empty set is the only initial object in $Set$.)
\end{definition}

\begin{lemma}
	Any two terminal objects are isomorphic. Same for any two initial objects.
\end{lemma}





\section{Functors and natural morphisms}

\begin{definition}
	A functor $F$ from a category $\cat{C}$ to a category $\cat{D}$ consists of operations $F_0: \cato{C} \rightarrow \cato{D}$ and $F_1: \catm{C} \rightarrow \catm{D}$ such that the following holds:
	\begin{itemize}
			\item for each $f: \ob{a} \rightarrow \ob{b}$ in $\cat{C}$: $F_1(f): F_0(\ob{a}) \rightarrow F_0(\ob{b})$
			\item for each
			\begin{tikzcd}
				\ob{a} \arrow[r, "f"] & \ob{b} \arrow[r, "g"] & \ob{c}
			\end{tikzcd}: $F_1(gf) = F_1(g)F_1(f)$.
			\item $F_0(id_{\ob{a}}) = id_{F_0(\ob{a})}$ for each $a \in \cat{C}$
	\end{itemize}
\end{definition}

\begin{definition}
	An endofunctor is a functor $F$ from a category $\cat{C}$ to itself.
\end{definition}

\begin{definition}
	A functor $F: \cat{C} \rightarrow \cat{D}$ is called \textbf{covariant}, whereas a functor $F: \catop{C} \rightarrow \cat{D}$ is called \textbf{contravariant}.
\end{definition}

Since functors are composable, the following definition can be made:

\begin{definition}
	$\scat$ is the category whose objects are \textbf{small categories} and whose morphisms are functors between those categories.
\end{definition}

Now that functors are defined, its time to define natural morphisms:

\begin{definition}
	Let $F, G: \cat{C} \rightarrow \cat{D}$ be two functors of categories $\cat{C}$ and $\cat{D}$: a \textbf{natural morphism} $\alpha: F \Rightarrow G$ is a set of morphisms
	$\alpha_{\ob{a}}: F(\ob{a}) \rightarrow G(\ob{a})$ for any object $\ob{a} \in \cato{C}$ such that for any arrow $f: \ob{a} \rightarrow \ob{b}$ in $\catm{C}$ the following diagram in $\cat{D}$ commutes:

	\begin{tikzcd}
		F(\ob{a}) \arrow[r, "F(f)"] \arrow[d, "\alpha_{\ob{a}}"] & F(\ob{b}) \arrow[d, "\alpha_{\ob{b}}"] \\
		G(\ob{a}) \arrow[r, "G(f)"] & G(\ob{b})
	\end{tikzcd}
\end{definition}

\begin{definition}
	A natural transformation $\alpha: F \Rightarrow G$ is called a \textbf{natural isomorphism} iff its components $\alpha_{\ob{a}}: \ob{a} \in \cat{C}$ are isomorphisms.
\end{definition}

\begin{definition}
	Let $\cat{C}$ and $\cat{D}$ be categories. We can define the \textbf{Functor category} which is denoted as either $\cat{D}^{\cat{C}}$ or $[\cat{C}, \cat{D}]$ as follows:
	\begin{itemize}
		\item objects of this category are functors $F: \cat{C} \rightarrow \cat{D}$
		\item morphisms between two such functors $F$ and $G$ are natural transformations $\alpha: F \Rightarrow G$
	\end{itemize}
\end{definition}

\begin{definition}
	Actually a category like $\scat$ which has morphisms and natural transformations between morphisms is called a \textbf{2-category}.
\end{definition}

\begin{definition}
	Let $\cat{C}$ and $\cat{D}$ be two categories, the \textbf{product category} or \textbf{cartesian product of categories} $\cat{C} \times \cat{D}$ has objects $(\ob{a}, \ob{b})$
	where $\ob{a} \in \cato{C}$ and $\ob{b} \in \cato{D}$ and morphisms $(f,b): (\ob{a}, \ob{b}) \rightarrow (\ob{a'}, \ob{b'})$ where $f: \ob{a} \rightarrow \ob{a'} \in \catm{C}$ and $g: \ob{b} \rightarrow \ob{b'} \in \catm{D}$.
\end{definition}

\begin{definition}
	A functor from any product category to another category is called a \textbf{bifunctor}.
\end{definition}

\begin{definition}
	A functor $F: \cat{C} \rightarrow \cat{D}$ is called \textbf{faithfull} if its induced function $F_1: Hom(\ob{a}, \ob{b}) \rightarrow Hom(F(\ob{a}), F(\ob{b}))$ is injective for any two objects $\ob{a}, \ob{b} \in \cat{C}$.
	It's called \textbf{full} if this induced function is surjective for any two objects $\ob{a}, \ob{b} \in \cat{C}$.
\end{definition}

\begin{definition}
	A functor \textbf{reflects} a property if, whenever the image (of objects or arrows) has a property, then the origin has the property too.
\end{definition}

\begin{lemma}
	Faithfull functors reflect epis and monos
\end{lemma}

\begin{lemma}
	Full and Faithfull functors reflect property of being initial or terminal object.
\end{lemma}

\begin{definition}
	Suppose we have categories $\cat{C}$, $\cat{D}$ and $\cat{E}$ together with functors:
	\begin{tikzcd}
		\cat{C} \arrow[bend left]{r}{F} \arrow[bend right]{r}[below]{F'} & \cat{D} \arrow[bend left]{r}{G} \arrow[bend right]{r}[below]{G'} & \cat{E}
	\end{tikzcd}
	If we have natural transformations $\alpha: F \Rightarrow F'$ and $\beta: G \Rightarrow G'$, then they cannot be trivially composed since their domains and codomains do not match.
	Writing out the inducing morphisms on objects, however, makes it easy to construct a morphism
	$$\beta \circ \alpha: G \circ F \Rightarrow G' \circ F'$$
	and this composition is called the \textbf{horizontal composition} of natural transformations.
\end{definition}


\section{limits and colimits}

\begin{definition}
	Let $F: \cat{C} \rightarrow \cat{D}$ be a functor between categories. A \textbf{cone} for $F$ exists of an object $\ob{d} \in \cato{D}$ and a natural transformation $\mu: \Delta_\ob{d} \Rightarrow F$.
	($\Delta_\ob{d}$ is the constant functor which sends all objects to $\ob{d}$ and all arrows to $id_{\ob{d}}$.)
	So $\mu$ is a family of morphisms $(\mu_\ob{c}: \ob{d} \rightarrow F(\ob{c})| \ob{c} \in \cato{C})$ such that for any morphism \\
	$f: \ob{c} \rightarrow \ob{c}'$ in $\cat{C}$ the following diagram commutes:
	\begin{tikzcd}
		& \ob{d} \arrow[ld, "\mu_\ob{c}" left, near start] \arrow[rd, "\mu_\ob{c}'"] \\
		F(\ob{c}) \arrow[rr, "F(f)"] & & F(\ob{c}')
	\end{tikzcd}
	This cone is denoted as $(\ob{d}, \mu)$ and $\ob{d}$ is called the \textbf{vertex} of the cone.
	In fact, cones of $F$ form a category with morphisms $f: (\ob{d}, \mu) \rightarrow (\ob{d}', \mu')$ being a morphism $f: \ob{d} \rightarrow \ob{d}'$ in $\catm{C}$ such that $\mu'_\ob{c}g = \mu_\ob{c}$
	for any $\ob{c} \in \cato{C}$. I.e. the following diagram commutes for any $\ob{c} \in \cato{C}$:
	\begin{tikzcd}
		\ob{d} \arrow[rr, "g"] \arrow[dr, "\mu_\ob{c}"] & & \ob{d}' \arrow[dl, "\mu_\ob{c}'"] \\
		& F(\ob{c})
	\end{tikzcd}
	So all cones over $F$ form a category, which is denoted $Cone(F)$.
\end{definition}

Note that the codomain of $F$ in the previous definition is regularly called the \textbf{diagram of $\cat{C}$ in $\cat{D}$} and the category $\cat{C}$ is called the \textbf{index category} of this diagram.

\begin{definition}
	Let $F$ be as in the previous definition. The \textbf{terminal object} in the category $Cone(F)$ is called the \textbf{limiting cone} for $F$ or \textbf{limit} for the diagram $F$. It is also sometimes denoted $\limc{F}$.
	Note that I say \textbf{the} terminal object because terminal objects are unique up to unique isomorphism.

\end{definition}

Now its natural to define the co-definition for cones and limits:

\begin{definition}
	Let $F: \cat{E} \rightarrow \cat{C}$ be a functor. A \textbf{cocone} for this functor is a pair $(\nu, \ob{d})$ where $\ob{d}$ is an object in $\cat{C}$ and $\nu: F \Rightarrow \Delta_\ob{d}$ is a natural transformation.
	So a morphism $f: (\nu, \ob{d}) \rightarrow (\nu', \ob{d}')$ in the category of cocones of $F$ are morphisms $f: \ob{d} \rightarrow \ob{d}'$ such that for any $\ob{e} \in \cato{E}$ the following diagram commutes:
	\begin{tikzcd}
		& F(\ob{c}) & \\
		\ob{d} \arrow[ru, "\nu_\ob{c}"] \arrow[rr, "f"] & & \ob{d}' \arrow[lu, "\mu_\ob{c}'"]
	\end{tikzcd}
	The \textbf{colimit} for $F$ is the \textbf{initial object} in the category of cocones over $F$.
\end{definition}

Now I will give a few examples of limits and colimits:

\begin{example}
	Let $\ob{c}, \ob{d}$ be two objects of a category $\cat{C}$. These two objects can be seen as the image of the functor $F: \bm{2} \rightarrow \cat{C}$ where $\bm{2}$ is the \textbf{discrete category with two objects}
	(the category with two objects and no arrows except for the identity arrows). The functor $F$ sends the one element of $\bm{2}$ to $\ob{c}$ and the other to $\ob{d}$.
	The \textbf{product} of $\ob{c}$ and $\ob{d}$ is the limit of the category of cones over $F$.
	I.e. it is an object $\ob{e}$ together with two morphisms $p: \ob{e} \rightarrow \ob{c}$ and $q: \ob{e} \rightarrow \ob{d}$ such that for any other such object $\ob{e}'$ and morphisms $p'$ and $q'$ there is a unique morphism $m$ that factorizes $p'$ and $q'$.
	In diagrams this looks as follows:
	\begin{tikzcd}
		& \ob{e}' \arrow[ddl, "p'" left, near start] \arrow{d}[description]{\exists ! m} \arrow[ddr, "q'"] \\
		& \ob{e} \arrow[dl, "p"] \arrow[dr, "q" left] \\
		\ob{c} & & \ob{d}
	\end{tikzcd}
	The morphisms $p$ and $q$ are called the \textbf{projections} of the product.
	Its \textbf{coproduct} of $\ob{c}$ and $\ob{d}$ is the colimit of this functor $F$. (Write out the diagram yourself.)
\end{example}

\begin{example}
	Let $\bar{\bm{2}}$ be the category
	\begin{tikzcd}
		%\ob{x} \rar[start anchor=real east, end anchor=real west]{a} \rar[start anchor=base east, end anchor=base west, swap]{b} & \ob{y}
		\ob{x} \arrow[bend left]{r}{a} \arrow[bend right]{r}[below]{b} & \ob{y}
	\end{tikzcd}.
	A functor $\bar{\bm{2}} \rightarrow \cat{C}$ is basically a parallel pair of morphisms $f,g: \ob{a} \rightarrow \ob{b}$ in $\cat{C}$. And a cone for this functor is basically a diagram
	\begin{tikzcd}
		& \ob{d} \arrow[ld, "\mu_\ob{a}" left, near start] \arrow[rd, "\mu_\ob{b}"] \\
		\ob{a} \arrow[bend left]{rr}{f} \arrow[bend right]{rr}[below]{g} & & \ob{b}
	\end{tikzcd}
	but $\mu_\ob{b} = f\mu_\ob{a} = g\mu_\ob{a}$ so this arrow is usually omitted, leaving the following diagram:
	\begin{tikzcd}
		\ob{d} \arrow[r, "\mu_\ob{a}"] & \ob{a} \arrow[bend left]{r}{f} \arrow[bend right]{r}[below]{g} & \ob{b}
	\end{tikzcd}.
	The limit in this category of cones (I.e. the object $\ob{d}$ and morphism $\mu_\ob{a}$) is called the \textbf{equalizer} of the pair $f$ and $g$.
	The definition of the \textbf{co equalizer} follows automatically.
\end{example}

\begin{lemma}
	Every equalizer is a monomorpism.

	Covariantly: every coequalizer is epi.
\end{lemma}

\begin{lemma}
	let \begin{tikzcd} \ob{d} \arrow[r, "\mu_\ob{a}"] & \ob{a} \arrow[bend left]{r}{f} \arrow[bend right]{r}[below]{g} & \ob{b} \end{tikzcd}
	be an equalizer diagram. Then $\mu_\ob{a}$ is an isomorphism iff $f=g$.

	Covariantly: a coequalizer is iso iff $f=g$.
\end{lemma}

\begin{definition}
	Every monomorphism $f: \ob{a} \rightarrow \ob{b}$ which is part of some equalizer diagram
	\begin{tikzcd} \ob{a} \arrow[r, "f"] & \ob{b} \arrow[bend left]{r}{g} \arrow[bend right]{r}[below]{h} & \ob{c} \end{tikzcd} is called \textbf{regular mono}.
	The definition of a \textbf{regular epi} follows automatically.
\end{definition}

\begin{lemma}
	If $f$ is regular mono and epi, it is iso. Also every split mono is regular.
\end{lemma}

\begin{example}
	A \textbf{pullback} is the limiting cone in the category of cones over a diagram like this:
	\begin{tikzcd}
		\ob{a} \arrow[r, "f"] & \ob{b} & \ob{c} \arrow[l, "g"]
	\end{tikzcd}.
	Just to clarify, a cone for this diagram looks like this:
	\begin{tikzcd}
		& \ob{d} \arrow[ld, "p"] \arrow[rd, "q"] \\
		\ob{a} \arrow[rd, "f"] & & \ob{c} \arrow[ld, "g"] \\
		& \ob{b}
	\end{tikzcd}.
	(The arrow from $\ob{d}$ to $\ob{b}$ is omitted since it can be given as the composition $gq = fp$.) The co definitions of the pullback is called a \textbf{pushout}.
\end{example}

\begin{lemma}
	In the pullback diagram given above, is $f$ is mono, then $q$ is too. Similarly if $f$ is iso, then $q$ is too.

	Co-Covariantly: let
	\begin{tikzcd}
		& \ob{b} \arrow[ld, "f"] \arrow[rd, "g"] \\
		\ob{a} \arrow[rd, "p"] & & \ob{c} \arrow[ld, "q"] \\
		& \ob{d}
	\end{tikzcd} be a pushout diagram, then $f$ epi implies that $q$ is epi and $f$ regular epi implies that $q$ is regular epi.
\end{lemma}

\begin{lemma}
	Given two commuting squares
	\begin{tikzcd}
		\ob{a} \arrow[r, "b"] \arrow[d, "a"] & \ob{b} \arrow[r, "c"] \arrow[d, "f"] & \ob{c} \arrow[d, "d"] \\
		\ob{x} \arrow[r, "g"] & \ob{y} \arrow[r, "h"] & \ob{z}
	\end{tikzcd}, then the following hold:
	\begin{itemize}
		\item If both squares are pullbacks, then so is the composite. Covariantly: the composition of two pushouts is a pushout.
		\item If the right hand square and the composite are pullbacks, then so is the left hand square.
		Covariantly: if both the first square and the composition are pushouts, then so is the second square.
	\end{itemize}
\end{lemma}

\begin{lemma}
	$f: \ob{a} \rightarrow \ob{b}$ is mono iff
	\begin{tikzcd}
		\ob{a} \arrow[r, "id_\ob{a}"] \arrow[d, "id_\ob{a}"] & \ob{a} \arrow[d, "f"] \\
		\ob{a} \arrow[r, "f"] & \ob{b}
	\end{tikzcd}
	is a pullback diagram.
	Covariantly: $f$ is epi iff
	\begin{tikzcd}
		\ob{a} \arrow[r, "f"] \arrow[d, "f"] & \ob{a} \arrow[d, "id_\ob{b}"] \\
		\ob{a} \arrow[r, "id_\ob{b}"] & \ob{b}
	\end{tikzcd} is a pushout diagram.
\end{lemma}



\section{yoneda}

\begin{definition}
	A \textbf{profunctor} is a bifunctor which is contravariant in its first argument and covariant in its second and which codomain is $Set$.
	So a profunctor is a functor $F: \catop{C} \times \cat{D} \rightarrow Set$.
\end{definition}

Let $\cat{C}$ be a locally small category and $\ob{a} \in \cato{C}$.
There is a functor $Hom(a, -): \cat{C} \rightarrow Set$ defined on objects by $\ob{x} \mapsto Hom(\ob{a}, \ob{x})$
and which sends arrows $f: \ob{x} \rightarrow \ob{y}$ to the function $Hom(a, f): Hom(\ob{a}, \ob{x}) \rightarrow Hom(\ob{a}, \ob{y})$	defined by $Hom(a, f)(h) := fh: \ob{a} \rightarrow \ob{y}$.

Similarly one can fix the second argument in the $Hom$ set in order to get a contravariant functor $Hom(-, b): \catop{C} \rightarrow Set$.

Combine these to functors in order to get a profunctor $Hom(-, -): \catop{C} \times \cat{C} \rightarrow Set$.

\begin{definition}
	Any functor $F: \cat{C} \rightarrow Set$ which is naturally isomorphic to the Hom functor $Hom(\ob{a}, -)$ for some object $\ob{a}$ is called \textbf{representable}.
\end{definition}

So this means that there are natural iso's $\alpha: Hom(\ob{a}, -) \rightarrow F$ and $\beta: F \rightarrow Hom(\ob{a}, -)$.

Naturality of $\alpha$ means that for all any arrow $f: \ob{x} \rightarrow \ob{y}$ the following diagram commutes: $$Ff \circ \alpha_\ob{x} = \alpha_\ob{y} \circ Hom(\ob{a}, f)$$.

($\alpha_\ob{x}$ is a function $Hom(\ob{a}, \ob{x}) \rightarrow F\ob{x}$.)

The following lemma is called the \textbf{yoneda lemma}:
\begin{lemma}
	Let $\cat{C}$ be any locally small category and furthermore let $\ob{a} \in \cato{C}$ and $F: \cat{C} \rightarrow Set$ be any functor such that $F\ob{a} \neq \emptyset)$:
	There is an isomorphism (of sets) $$[\cat{C}, Set](Hom(\ob{a}, -), F) \cong Fa$$
	(remember that $[\cat{C}, Set]$ is the category with functors $\cat{C} \rightarrow Set$ as objects and natural transformations as arrows.)
\end{lemma}

On one side, lets take any natural transformation $\alpha: Hom(\ob{a}, -) \rightarrow F$, then $\alpha_\ob{a}: Hom(\ob{a}, \ob{a}) \rightarrow Fa$ is a function of sets, which can be evaluated at
$id_\ob{a} \in Hom(\ob{a}, \ob{a})$ to get an element in $Fa$.
This way we defined a function $[\cat{C}, Set](Hom(\ob{a}, -), F) \rightarrow Fa$.

Its inverse is a function $Fa \rightarrow [\cat{C}, Set](Hom(\ob{a}, -), F)$ which can be constructed as follows:

Take any object $\ob{x}$ in $\cat{C}$, some morphism $h: \ob{a} \rightarrow \ob{x}$ and a morphism $f: \ob{x} \rightarrow \ob{y}$.
The naturality square for any natural morphism $\alpha$ given above becomes (point-wise) when acting on $h$:
$$\alpha_\ob{y}(Hom(\ob{a}, f)h) = (Ff)(\alpha_\ob{x}h)$$
and $$Hom(\alpha, f)h = f \circ h$$
which leads to:
$$\alpha_\ob{y}(f \circ h) = (Ff)(\alpha_\ob{x}h)$$

So specializing $\ob{x} = \ob{a}$ and $h = id_\ob{a}$ this yields:
$$\alpha_\ob{y}f = (Ff)(\alpha_\ob{a}id_\ob{a})$$
The left hand side is the action of $\alpha_\ob{y}$ on an arbitrary element $f \in Hom(\ob{a}, \ob{y})$ which it is completely determined by $\alpha_\ob{a}id_\ob{a}$.
So a natural transformation $\alpha$ can be determined uniquely by choosing any value in $F_\ob{a}$ and setting this as the value for $\alpha_\ob{a} id_\ob{a}$.

\textbf{Co-yoneda} lemma states that

\begin{lemma}
	$[\catop{C}, Set](Hom(-, \ob{a}), F) \cong F_\ob{a}$.
\end{lemma}

So using the Yoneda lemma and setting $F := Hom(\ob{b}, -)$ for any object $\ob{b}$ we get the following equality:
$$[\cat{C}, Set](Hom(\ob{a}, -), Hom(\ob{b}, -)) \cong Hom(\ob{b}, \ob{a})$$

\begin{definition}
	Define a function $h: \cat{C} \rightarrow Set^{\catop{C}}$ which sends $\ob{a}$ to the functor $h\ob{a} := Hom(\ob{a}, -): \catop{C} \rightarrow Set$ which is an element of $Set^\cat{C}$.
	This function can be made into a functor be defining its action on a function $f: \ob{b} \rightarrow \ob{a}$. $hf \in [\cat{C}, Set](Hom(\ob{a}, -), Hom(\ob{b}, -))$ using the previously given isomorphism.
	So the function is actually a functor $h: \cat{C} \rightarrow Set^{\catop{C}}$ which is called the \textbf{Yoneda embedding}.
\end{definition}

\begin{lemma}
	The Yoneda embedding is full, faithfull and injective on objects.
\end{lemma}

The use of the Yoneda lemma is often the following:
Suppose we want to prove that two objects $\ob{a}, \ob{b} \in \cato{C}$ are isomorphic.
It suffices to prove that for any object $\ob{x} \in \cato{C}$ there is a bijection $f_\ob{x}: Hom(\ob{x}, \ob{a}) \rightarrow Hom(\ob{x}, \ob{b})$ which is natural in $\ob{x}$.
Naturality in $\ob{x}$ means the for any morphism $g: \ob{x}' \rightarrow \ob{x}$ the following diagram commutes:
\begin{tikzcd}
	Hom(\ob{x}, \ob{a}) \arrow[r, "f_\ob{x}"] \arrow[d, "{Hom(g, id_\ob{a})}"] & Hom(\ob{x}, \ob{b}) \arrow[d, "{Hom(g, id_\ob{b})}"] \\
	Hom(\ob{x}', \ob{a}) \arrow[r, "f_\ob{x}'"] & Hom(\ob{x}', \ob{b})
\end{tikzcd}



\section{monads, adjunctions and T-algebras}

\begin{definition}
	Let $\cat{C}$ and $\cat{D}$ be categories with functors $R: \cat{C} \rightarrow \cat{D}$ and $L: \cat{D} \rightarrow \cat{C}$.
	$R$ is called \textbf{right adjoint} to the functor $L$ (And conversely $L$ \textbf{left adjoint} to $R$) if there are natural transformations
	$$\eta: I_\cat{D} \Rightarrow R \circ L$$
	$$\epsilon: L \circ R \Rightarrow I_\cat{C}$$
	where $I_\cat{D}$ and $I_\cat{C}$ are the identity functors for the categories $\cat{D}$ and $\cat{C}$ respectively.
	The whole thing is called an \textbf{adjunction} and is usually denoted:
	$$L \dashv R$$
	In this definition $\eta$ is called the \textbf{unit} and $\epsilon$ the \textbf{counit} of the adjunction.
\end{definition}

Note that on objects these natural transformations become moprhisms
$$\eta_\ob{d}: \ob{d} \rightarrow (R \circ L)(\ob{d})$$
$$\epsilon_\ob{c}: (L \circ R)(c) \rightarrow \ob{c}$$
for objects $\ob{d} \in \catop{D}$ and $\ob{c} \in \catop{C}$.

\begin{lemma}
	The unit and counit adhere to the following \textbf{triangular identities}:
	\begin{tikzcd}
		L \arrow[r, "L \circ \eta"] \arrow[rd, equal] & L \circ R \circ L \arrow[d, "\epsilon \circ L"] \\
		& L
	\end{tikzcd}
	\begin{tikzcd}
		R \arrow[r, "\eta \circ R"] \arrow[rd, equal] & R \circ L \circ R \arrow[d, "R \circ \epsilon"] \\
		& R
	\end{tikzcd}
	Here the compositions $L \circ \eta$, $\epsilon \circ L$, $\eta \circ R$ and $R \circ \epsilon$ are horizontal compositions or natural transformations.
	(Where $L = I_L: L \Rightarrow L$ is the identity natural transformation from $l$ to $L$ and $R$ similar.)
\end{lemma}

Alternatively an adjunction can be described using hom sets:

\begin{definition}
	Let $R: \cat{C} \rightarrow \cat{D}$ and $L: \cat{D} \rightarrow \cat{C}$ be functors. $L$ is left adjoint to $R$ iff for objects $\ob{c} \in \cat{C}$
	and $\ob{d} \in \cat{D}$ there is an isomorphism $$ Hom(L\ob{d}, \ob{c}) \cong Hom(\ob{d}, R\ob{c})$$
	which is natural in both $\ob{c}$ and $\ob{d}$.
\end{definition}

\begin{lemma}
	There is a \textbf{forgetfull functor} $U$ from $Mon$ (the category of monoids) to $Set$ and conversely there is a \textbf{free functor}
	$F: Set \rightarrow Mon$ which maps a set to the monoid which is freely generated by the set.
	For this situation, $F$ is left adjoint to $U$.

	The same goes for the forgetfull functor $U: Grp \rightarrow Set$ which maps a group to its underlying set and $F: Set \rightarrow Grp$ which maps a set to the group which is freely generated by this set.
\end{lemma}

\begin{definition}
	Let $\cat{C}$ be a category and $T: \cat{C} \rightarrow \cat{C}$ be any endofunctor.
	The \textbf{kleisli category} associated to $\cat{C}, T$ is denoted $\cat{C}_T$.
	It has the same objects as $\cat{C}$, but an arrow $\ob{a} \rightarrow \ob{b}$ in $\cat{C}_T$ is actually an arrow $\ob{a} \rightarrow T\ob{b}$ in $\cat{C}$.
	Arrows can be composed if $T$ is a monad, as will be described shortly.
\end{definition}

\begin{definition}
	A \textbf{monad} is an endofunctor $T$ together together with two natural transformations
	$$ \mu: T^2 \Rightarrow T $$
	and
	$$ \eta: I \Rightarrow T $$
	which make the following two diagrams commute:

	\begin{tikzcd}
		T^3 \arrow[r, "\mu \circ T"] \arrow[d, "T \circ \mu"] & T^2 \arrow[d, "\mu"] \\
		T^2 \arrow[r, "\mu"] & T
	\end{tikzcd}
	and
	\begin{tikzcd}
		I \circ T \arrow[r, "\eta \circ T"] \arrow[rd] & T^2 \arrow[d, "\mu"] & T \circ I \arrow[l, "T \circ \eta"] \arrow[ld] \\
		& T
	\end{tikzcd}
	Note that again $\mu \circ T$, $T \circ \mu$, $\eta \circ T$ and $T \circ \eta$ are horizontal compositions of natural transformations.
	$\mu$ is usually called \textbf{multiplication} and $\eta$ \textbf{unit}.
\end{definition}

\begin{lemma}
	Let $\cat{C}$ be a category and $(T, \eta, \mu)$ a monad on $\cat{C}$, then the kleisli category $\cat{C}_T$ is defined and the composition of arrows are as follows:
	If $f: \ob{a} \rightarrow T\ob{b}, g: \ob{b} \rightarrow T\ob{c} \in \catm{C}$ (So $f: \ob{a} \rightarrow \ob{b}, g: \ob{b} \rightarrow \ob{c}$ in $(\cat{C_T})_1$.)
	then $g \circ_T f = \mu \circ Tg \circ f$, is the kleisli composition of arrows. (Note that $g \circ_T f: a \rightarrow T\ob{c}$ in $\catm{C}$.)
\end{lemma}

Ofcourse, it's also possible to define a comonad:

\begin{definition}
	A \textbf{comonad} on a category $\cat{C}$ is an endofunctor $T: \cat{C} \rightarrow \cat{C}$ together with natural transformations
	$$ \epsilon: T \Rightarrow I $$
	and
	$$ \delta: T \Rightarrow T^2 $$
	called \textbf{extract} and \textbf{duplicate} respectively.
	These natural transformations should make the following two diagrams commute:

	\begin{tikzcd}
		T \arrow[r, "\delta"] \arrow[d, "\delta"] & T^2 \arrow[d, "\delta \circ T"] \\
		T^2 \arrow[r, "T \circ \delta"] & T^3
	\end{tikzcd}
	and
	\begin{tikzcd}
		I \circ T & T^2 \arrow[l, "\epsilon \circ T"] \arrow[r, "T \circ \epsilon"] & T \circ I \\
		& T \arrow[lu] \arrow[u, "\delta"] \arrow[ru]
	\end{tikzcd}
\end{definition}

The following theorem shows that every adjunction gives rise to a monad:

\begin{theorem}
	Let $L \dashv R$ be an adjunction. So we have natural transformations $$\eta: I_\cat{D} \Rightarrow R \circ L$$ and $$ \epsilon: L \circ R \Rightarrow I_\cat{C} $$.
	Then $R \circ L$ is a monad with unit $\eta$ and multiplication $\mu = R \circ \epsilon \circ L$.

	Similarly $L \circ R$ forms a comonad with $\epsilon$ as extract and $\delta = L \circ \eta \circ R: (L \circ R)^2 \rightarrow L \circ R$.
\end{theorem}

\begin{definition}
	Let $\cat{C}$ be a category with endofunctor $T$. We will now define the \textbf{category of $T$-algebras}.
	Objects in this category are pairs $(\ob{a}, f)$, where $\ob{a} \in \cato{C}$ and $f: T\ob{a} \rightarrow \ob{a}$ in $\catm{C}$.
	Here $\ob{a}$ is often called the \textbf{carrier} and $f$ is called the \textbf{evaluator function}.
	A morphism $\mu: (\ob{a}, f) \rightarrow (\ob{b}, g)$ is simply a morphism $\mu: \ob{a} \rightarrow \ob{b}$ in $\catm{C}$ that makes the following diagram commute:

	\begin{tikzcd}
		T\ob{a} \arrow[r, "Tm"] \arrow[d, "f"] & T\ob{b} \arrow[d, "g"] \\
		\ob{a} \arrow[r, "m"] & \ob{b}
	\end{tikzcd}

	Any object in this category is called a \textbf{$T$-albegra}.
	An initial object in this category is called the \textbf{initial algebra}.
\end{definition}

\begin{lemma}
	Let $(\ob{i}, j)$ be an initial algebra, then the morphism $j: T\ob{i} \rightarrow \ob{i}$ is an isomorphism.
\end{lemma}

\begin{definition}
	Let $\cat{C}$ be a category with endofunctor $T$. Objects in the \textbf{category of $T$-coalgebras} are pairs $(\ob{a}, f)$ where $\ob{a} \in \cato{C}$ and $f: \ob{a} \rightarrow T\ob{a}$.
	And a morphism $m: (\ob{a}, f) \rightarrow (\ob{b}, g)$ in this category is a morphism $m: \ob{a} \rightarrow \ob{b}$ which makes the following diagram commute:

	\begin{tikzcd}
		T\ob{a} \arrow[r, "Tm"] & T\ob{b} \\
		\ob{a} \arrow[u, "f"] \arrow[r, "m"] & \ob{b} \arrow[u, "g"]
	\end{tikzcd}
\end{definition}

\begin{lemma}
	The arrow $u$ in a terminal objet $(\ob{t}, u)$ of a $T$-coalgebra is an isomorphism.
\end{lemma}

\begin{definition}
	Let $\cat{C}$ be a category, $T$ an endofunctor and $(\ob{a},f)$ the initial $T$-algebra. (So $\ob{a} \in \cato{C}$ and $f: T\ob{a} \rightarrow \ob{a}$ is iso.)
	Let furthermore $(\ob{b}, g)$ be any other object in the category of $T$-algebras. Then, since $(\ob{a}, f)$ is the initial object, there is a unique arrow $m: (\ob{a}, f) \rightarrow (\ob{b}, g)$ in the category of $T$-algebras.
	So there is a unique $m: \ob{a} \rightarrow \ob{b}$ that makes the following square commute:
	\begin{tikzcd}
		T\ob{a} \arrow[r, "Tm"] \arrow[d, "f"] & T\ob{b} \arrow[d, "g"] \\
		\ob{a} \arrow[r, "m"] & \ob{b}
	\end{tikzcd}.
	This uniquely defined morphism $m$ is called the \textbf{catamorphism of $g$}, usually denoted $\cata{g}$.
\end{definition}

Note that, since $f$ is iso, $\cata{g}$ can also be defined as the composition $\cata{g} = g \circ Tm \circ f^{-1}$.

So this equality actually gives a recursive definition of $m = \cata{g}$.

It's already been shown that every adjunction gives rise to a monad. What follows next is the opposite.
I will show that every monad gives rise to a an adjunction.

\begin{definition}
	Let $\cat{C}$ be a category and $T$ a monad. Since $T$ is an endofunctor, we can define the category of $T$-algebras.
	This category is called the \textbf{Eilenberg-Moore category} and is usually denoted by $\cat{C}^T$.
\end{definition}

\begin{theorem}
	For such a $T$-algebra there is clearly a forgetful functor $U^T: \cat{C}^T \rightarrow \cat{C}$ which sends objects $(\ob{a}, f)$ to the carrier object $\ob{a}$
	and morphisms of the $T$-algebra to the corresponding morphism in $\catm{C}$.

	As noted before a forgetful functor is left adjoint to a free functor $F^T: \cat{C} \rightarrow \cat{C}^T$.
	This free functor sends an object $\ob{a} \in \cato{C}$ to the object $(T\ob{a}, \mu_\ob{a})$, where $\mu$ is the multiplication of the monad $T$.
\end{theorem}

It even turns out that
$$U^T \circ F^T = T$$
as one might hope.

\begin{theorem}
	An \textbf{applicative functor} is an endofunctor $F: \cat{C} \rightarrow \cat{C}$ together with:
	\begin{itemize}
		\item a natural morphism $I \rightarrow F$ called \textbf{pure}. (Think of the unit of a monad which sends objects $\ob{a}$ to $F(\ob{a})$ and morphisms $f: \ob{a} \rightarrow \ob{b}$ to a morphism $pure(f): F(\ob{a}) \rightarrow F(\ob{b})$.)
		\item a binary operation $\circledast: F(\ob{a} \rightarrow \ob{b}) \rightarrow F(\ob{a}) \rightarrow F(\ob{b})$. \newline
			  (You can think of $\circledast$ as a binary operation that sends the lift of a function $f: \ob{a} \rightarrow \ob{b}$ and the lift of an element $x: \ob{a}$ to $f \circledast x = F(f)(F(\ob{a})): F(\ob{b})$.) \newline
		      So in a way the function $F(\ob{a} \rightarrow \ob{b}) \circledast F(\ob{a})$ is the \textbf{apply} function on lifted functions $f$.
	\end{itemize}
	In order for this definition to be an applicative functor, the following four conditions must hold:
	\begin{itemize}
		\item \textbf{Identity law}: $pure(id) \circledast x = x$. This means that $pure(id_\ob{a}) \circledast \_ = id_{F(\ob{a})}$.
		\item \textbf{Homomorphism law}: $pure(\circ) \circledast u \circledast v \circledast w = u \circledast v \circledast w$
		      where $u: \ob{a} \rightarrow \ob{b}$, $v: \ob{b} \rightarrow \ob{c}$, $w: \ob{a} \rightarrow \ob{c}$ and $\circ: (\ob{a} \rightarrow \ob{b}, \ob{b} \rightarrow \ob{c}) \rightarrow (\ob{a} \rightarrow \ob{c})$ is function composition.
		      So this basically says that wrapped composition applies as composition.
		\item \textbf{Composition law}: $pure(f) \circledast pure(x) = pure(f(x))$.
		      Which means that applying lifted function to lifted value is the same as lifting the result of applying the function to the element.
		\item \textbf{Interchange law}: $f \circledast pure(x) = pure(g \mapsto g(x)) \circledast f$ where $f: F(\ob{a}) \rightarrow F(\ob{b})$, $x \in \ob{A}$ and thus $g \mapsto g(x): F(\ob{a} \rightarrow \ob{b}) \rightarrow F(\ob{b})$.
		      This means that one can change the order of the apply's method.
	\end{itemize}
\end{theorem}

\begin{example}
	The functor $List$ is an applicative functor.
	In this example the pure function maps an element $a$ to $List(a)$. The binary operator $\circledast$ is more interesting:
	let $L_f = List(f_0, \dots, f_n)$ be a list of functions $f_i: \ob{a} \rightarrow \ob{b}$ and let $L_a = List(a_0, \dots, a_m)$ be a list of elements $a_i \in \ob{a}$, then
	$$L_f \circledast L_a = List(f_0(a_0), \dots, f_0(a_m), f_1(a_0), \dots, f_1(a_m), \dots, f_n(a_0), \dots, f_n(a_m))$$
\end{example}

\end{document}